\documentclass[12pt, letterpaper]{article}
\usepackage{graphicx} % Required for inserting images
\usepackage{hyperref}
\usepackage{listings}
\usepackage{amssymb}
\usepackage{amsmath}
\usepackage[english]{babel}
\usepackage{nicefrac, xfrac}
\usepackage{mathtools}
\newcommand{\acc}{\\\hphantom{}\\}
\usepackage[dvipsnames]{xcolor}
\usepackage[table,xcdraw]{xcolor}
\definecolor{light-gray}{gray}{0.95}
\newcommand{\code}[1]{\colorbox{light-gray}{\texttt{#1}}}
\newcommand{\codee}[1]{\colorbox{white}{\texttt{#1}}}
\usepackage[paper=a4paper,left=20mm,right=20mm,bottom=25mm,top=25mm]{geometry}
\renewcommand{\labelenumii}{\arabic{enumi}.\arabic{enumii}}
\renewcommand{\labelenumiii}{\arabic{enumi}.\arabic{enumii}.\arabic{enumiii}}
\renewcommand{\labelenumiv}{\arabic{enumi}.\arabic{enumii}.\arabic{enumiii}.\arabic{enumiv}}
\newcommand{\id}{{\hphantom{ident}}}
\newcommand{\vincolo}[1]{\colorbox{Orange}{$[$\text{#1}$]$}}
\title{\textbf{eBuy}}

\date{}


\begin{document}

\maketitle\section{Requisiti}
\hphantom{a}\\
1.  \textbf{Utente}\\
\id	1.1 nome\\
\id	1.2 data registrazione\\
\id	1.3 bid fatte\\
\acc
2.  \textbf{Post} (annuncio)\\
\id2.1 descrizione oggetto\\
\id2.2 categoria (fai classe categoria)\\
\id	2.3 garanziaInAnni int$\ge$0\\
\id	2.4 pagabile in (implementato con enum)\\
\id\id		2.4.1 bonifico\\
\id\id		2.4.2 carta di credito\\
\id	2.5 usato? (implementato con enum)\\
\id\id		2.5.1 nuovo\\
\id\id\id			2.5.1.1 garanziaInAnni int$>$1 (attributo specializzato)\\
\id\id	2.5.2 usato\\
\id\id\id		2.5.2.1 condizioni {ottimo,buono,discreto, da sistemare}\\
\id2.6 asta o no? (disjoint.complete)\\
\id\id		2.6.1 post con asta\\
\id\id\id		2.6.1.1 prezzo iniziale in (euro,centesimi)\\
\id\id\id		2.6.1.2 prezzo rialzi in (euro,centesimi)\\
\id\id\id		2.6.1.3 istante scadenza asta (date time)\\
\id\id\id		2.6.1.4 insieme di Bid (offerte)\\
\id\id	2.6.2 post senza asta\\
        \id\id\id		2.6.2.1 prezzo in (euro,centesimi)\\
        \id\id\id		2.6.2.1 utente che ha effettuato l'acquisto\\
\acc
3. \textbf{Bid} (offerta)\\
\id	3.1 Utente che ha fatto l'offerta\\
\id	3.2 Post in questione\\
    \id	3.3 Istante offerta \\
    \id	3.4 ordine nell'offerta, n se è l'n-esimo utente che fa l'offerta\\
    \id	3.5 prezzo offerta = n$\cdot$Post.rialzo+prezzo iniziale\\
    \id	3.6 bid vincente o no?\\
\section{Diagramma UML}\begin{center}
    \includegraphics[width=\textwidth ]{images/UML.eps}
\end{center}
\newpage
\section{Diagramma Use-Case}\begin{center}
    \includegraphics[width=\textwidth ]{images/UseCase.eps}
\end{center}
\newpage
\section{Specifiche}
\subsection{Specifica dei tipi di dato}
Price = (euro : Int$>$0, cent : $[$0..99$]$ ) 
\subsection{Specifica delle classi}
\subsubsection{Bid}
\code{numero\_bid () : Intero}\begin{itemize}
    \item \textit{pre-condizioni} : Nessuna
    \item \textit{post-condizioni} : Non modifica il livello degli oggetti.\\ 
    Sia $a:Asta$ l'oggetto per cui esiste il link ($this,a$)\\ 
    Sia $B$ l'insieme di tutti gli oggetti $x:Bid$ per cui $\exists (x,a) \land x.istante<this.istante$\\ 
    $result = |B|+1$ 
\end{itemize}
\subsection{Specifica dei vincoli esterni}
\vincolo{V.Bid.istante\_offerta} : $\forall b:Bid$  e $\forall a:Asta$ per cui $\exists (a,b)$, deve 
essere vero che $b.istante\le a.scadenza$.\acc
\vincolo{V.Bid.istante\_reg\_utente} :  $\forall b:Bid$  e $\forall u:Utente$ per cui $\exists (u,b)$, deve 
essere vero che $b.istante\ge u.registrazione$.\acc
\vincolo{V.Utente.scadenza\_aste} : $\forall u:Utente$, sia $P$ l'insieme degli oggetti $p:Asta$ tale che 
$\exists (p,u):vende$. $\forall p\in P$ deve essere vero che $p.scadenza\ge u.registrazione$.\acc 
\newpage
\subsection{Specifica degli use-case}
\subsubsection{Utente}
\code{offerta (a:Asta, u:Utente) :  Bid }\begin{itemize}
    \item \textit{pre-condizioni} : Non deve esistere $(u,a):pubblica$. 
    \item \textit{post-condizioni} : Viene creato un oggetto $b:Bid$ tale che:\\ 
     $b.istante=now$ \\
     $\exists (u,b) : offerta$ \\
     $\exists (a,b) $ \\ 
     Sia $r=a.rialzi$\\ 
     Sia $price=a.prezzo$\\
     $b.prezzo = price + r\cdot ($\codee{this.numero\_bid()} $-1 )$\\
     Viene creato $(u,a):acquista/bidder$ 
\end{itemize}
\code{compraOggetto (c:CompraSubito, u:Utente) }\begin{itemize}
    \item \textit{pre-condizioni} : Non deve esistere $(u,c):pubblica$, non deve 
    esistere un link di tipo $acquista$ in cui è coinvolto $c$. 
    \item \textit{post-condizioni} : Viene creato un link di tipo $(u,c):scquista/bidder$.
\end{itemize}
\code{ creaAstaNuovo ( u : Utente, prezzoIniziale : Price, desc :Stringa, gar : Intero$>$1,
}\\\code{ pag : \{Bonifico,Carta\}[1..2],
rialzi : Price, scad : DateTime$> now$,}\\\code{ cat : Categoria[1..*])}\begin{itemize}
    \item \textit{pre-condizioni} : Deve esistere almeno un oggetto di tipo Categoria.
    \item \textit{post-condizioni} : Viene creato un oggetto $a:Asta$ tale che\\ 
    $a.Price = prezzoIniziale$\\ 
    $a.descrizione\_oggetto = desc$\\ 
    $a.garanzia\_in\_anni = gar$\\ 
    $a.pagamento = pag$\\
    $\forall c \in cat$, crea un link $(a,c)$\\
    $a.scadenza=sca$ \\
    $a.rialzi=rialzi$\\
    Viene creato un link $(u,a):pubblica$.
\end{itemize}
Simile ed analogo per i metodi realativi al:\begin{itemize}
    \item Creare un post (compra subito) per un oggetto usato 
    \item Creare un post (compra subito) per un oggetto nuovo 
    \item Creare un asta per un oggetto usato
\end{itemize}






\end{document}